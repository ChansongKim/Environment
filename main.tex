\documentclass[12pt]{article}

\usepackage{titletoc}
\usepackage[colorlinks=true,citecolor=red!50!black,urlcolor=blue!50!black,linkcolor=red!50!black]{hyperref}
\usepackage{chngcntr}
\usepackage{amsthm}

\author{
Nicholas R. Davis\footnote{Research Director, Argosy Foundation,  \href{mailto:nick@argosyfnd.org}{nick@argosyfnd.org}}\and
Shin Young Park\footnote{PhD Candidate, University of Wisconsin-Milwaukee}\and
Chan Song Kim\footnote{PhD Student, University of Wisconsin-Milwaukee}\and
Patrick W. Kraft\footnote{Assistant Professor, University of Wisconsin-Milwaukee, \href{mailto:kraftp@uwm.edu}{kraftp@uwm.edu}}
}

\title{Don't Shoot the Messenger\\
\large{The Role of Expert Cues in Correcting Misinformation on Climate Change}}
\date{April 7, 2022}

% fancy font
\usepackage{mathpazo}

\begin{document}
\maketitle
\thispagestyle{empty}

\begin{center}
-- WORK IN PROGRESS -- \\
PLEASE DO NOT CITE OR REDISTRIBUTE WITHOUT PERMISSION
\end{center} 

\begin{abstract}\singlespacing
\noindent Despite scientific consensus about the imminent existential threat of human-induced climate change, misinformation on its cause, extent, and timeline is still common. Previous research has found that partisan bias is a major motivation to produce and spread misinformation, especially for such politically polarized issues. Thus, manipulation of partisan cues to correct falsehoods has been the subject of prior experimental work. However, non-partisan "expert" cues are a common source of factual information on climate change. Yet, there has been less attention to understanding how expert cues can reduce misinformation. We propose that factual expert cues can combat misinformation on politically contentious issues like climate change. We contend that experts delivering messages are more effective than impersonally providing facts, even without the co-partisan activation mechanism scrutinized in past research. Using a preregistered survey experiment, we present subjects with four expert cues: journalist, city manager, scientist, and activist. We expect that these expert cues will perform two functions to reduce misinformation about climate change: update respondents' beliefs with factual information, and increase the willingness of respondents to share the factual information. The influence of each expert cue can vary depending on an individual's prior belief and trust in the source cue. Therefore, we also examine how the expert's identities interact with race and gender.

\vspace{\baselineskip}
\noindent \textit{Keywords}: misinformation, elite cue, climate change, survey experiment 

\vspace{\baselineskip}
%\noindent \textit{Word count}: 5873 (via TeXcount)
\end{abstract}
\hfill

% \onehalfspacing
\doublespacing

\newpage\setcounter{page}{1}
\documentclass[12pt]{article}

\usepackage{titletoc}
\usepackage[colorlinks=true,citecolor=red!50!black,urlcolor=blue!50!black,linkcolor=red!50!black]{hyperref}
\usepackage{chngcntr}
\usepackage{amsthm}

\author{
Nicholas R. Davis\footnote{Research Director, Argosy Foundation,  \href{mailto:nick@argosyfnd.org}{nick@argosyfnd.org}}\and
Shin Young Park\footnote{PhD Candidate, University of Wisconsin-Milwaukee}\and
Chan Song Kim\footnote{PhD Student, University of Wisconsin-Milwaukee}\and
Patrick W. Kraft\footnote{Assistant Professor, University of Wisconsin-Milwaukee, \href{mailto:kraftp@uwm.edu}{kraftp@uwm.edu}}
}

\title{Don't Shoot the Messenger\\
\large{The Role of Expert Cues in Correcting Misinformation on Climate Change}}
\date{April 7, 2022}

% fancy font
\usepackage{mathpazo}

\begin{document}
\maketitle
\thispagestyle{empty}

\begin{center}
-- WORK IN PROGRESS -- \\
PLEASE DO NOT CITE OR REDISTRIBUTE WITHOUT PERMISSION
\end{center} 

\begin{abstract}\singlespacing
\noindent Despite scientific consensus about the imminent existential threat of human-induced climate change, misinformation on its cause, extent, and timeline is still common. Previous research has found that partisan bias is a major motivation to produce and spread misinformation, especially for such politically polarized issues. Thus, manipulation of partisan cues to correct falsehoods has been the subject of prior experimental work. However, non-partisan "expert" cues are a common source of factual information on climate change. Yet, there has been less attention to understanding how expert cues can reduce misinformation. We propose that factual expert cues can combat misinformation on politically contentious issues like climate change. We contend that experts delivering messages are more effective than impersonally providing facts, even without the co-partisan activation mechanism scrutinized in past research. Using a preregistered survey experiment, we present subjects with four expert cues: journalist, city manager, scientist, and activist. We expect that these expert cues will perform two functions to reduce misinformation about climate change: update respondents' beliefs with factual information, and increase the willingness of respondents to share the factual information. The influence of each expert cue can vary depending on an individual's prior belief and trust in the source cue. Therefore, we also examine how the expert's identities interact with race and gender.

\vspace{\baselineskip}
\noindent \textit{Keywords}: misinformation, elite cue, climate change, survey experiment 

\vspace{\baselineskip}
%\noindent \textit{Word count}: 5873 (via TeXcount)
\end{abstract}
\hfill

% \onehalfspacing
\doublespacing

\newpage\setcounter{page}{1}
\documentclass[12pt]{article}

\usepackage{titletoc}
\usepackage[colorlinks=true,citecolor=red!50!black,urlcolor=blue!50!black,linkcolor=red!50!black]{hyperref}
\usepackage{chngcntr}
\usepackage{amsthm}

\author{
Nicholas R. Davis\footnote{Research Director, Argosy Foundation,  \href{mailto:nick@argosyfnd.org}{nick@argosyfnd.org}}\and
Shin Young Park\footnote{PhD Candidate, University of Wisconsin-Milwaukee}\and
Chan Song Kim\footnote{PhD Student, University of Wisconsin-Milwaukee}\and
Patrick W. Kraft\footnote{Assistant Professor, University of Wisconsin-Milwaukee, \href{mailto:kraftp@uwm.edu}{kraftp@uwm.edu}}
}

\title{Don't Shoot the Messenger\\
\large{The Role of Expert Cues in Correcting Misinformation on Climate Change}}
\date{April 7, 2022}

% fancy font
\usepackage{mathpazo}

\begin{document}
\maketitle
\thispagestyle{empty}

\begin{center}
-- WORK IN PROGRESS -- \\
PLEASE DO NOT CITE OR REDISTRIBUTE WITHOUT PERMISSION
\end{center} 

\begin{abstract}\singlespacing
\noindent Despite scientific consensus about the imminent existential threat of human-induced climate change, misinformation on its cause, extent, and timeline is still common. Previous research has found that partisan bias is a major motivation to produce and spread misinformation, especially for such politically polarized issues. Thus, manipulation of partisan cues to correct falsehoods has been the subject of prior experimental work. However, non-partisan "expert" cues are a common source of factual information on climate change. Yet, there has been less attention to understanding how expert cues can reduce misinformation. We propose that factual expert cues can combat misinformation on politically contentious issues like climate change. We contend that experts delivering messages are more effective than impersonally providing facts, even without the co-partisan activation mechanism scrutinized in past research. Using a preregistered survey experiment, we present subjects with four expert cues: journalist, city manager, scientist, and activist. We expect that these expert cues will perform two functions to reduce misinformation about climate change: update respondents' beliefs with factual information, and increase the willingness of respondents to share the factual information. The influence of each expert cue can vary depending on an individual's prior belief and trust in the source cue. Therefore, we also examine how the expert's identities interact with race and gender.

\vspace{\baselineskip}
\noindent \textit{Keywords}: misinformation, elite cue, climate change, survey experiment 

\vspace{\baselineskip}
%\noindent \textit{Word count}: 5873 (via TeXcount)
\end{abstract}
\hfill

% \onehalfspacing
\doublespacing

\newpage\setcounter{page}{1}
\documentclass[12pt]{article}

\usepackage{titletoc}
\usepackage[colorlinks=true,citecolor=red!50!black,urlcolor=blue!50!black,linkcolor=red!50!black]{hyperref}
\usepackage{chngcntr}
\usepackage{amsthm}

\author{
Nicholas R. Davis\footnote{Research Director, Argosy Foundation,  \href{mailto:nick@argosyfnd.org}{nick@argosyfnd.org}}\and
Shin Young Park\footnote{PhD Candidate, University of Wisconsin-Milwaukee}\and
Chan Song Kim\footnote{PhD Student, University of Wisconsin-Milwaukee}\and
Patrick W. Kraft\footnote{Assistant Professor, University of Wisconsin-Milwaukee, \href{mailto:kraftp@uwm.edu}{kraftp@uwm.edu}}
}

\title{Don't Shoot the Messenger\\
\large{The Role of Expert Cues in Correcting Misinformation on Climate Change}}
\date{April 7, 2022}

% fancy font
\usepackage{mathpazo}

\begin{document}
\maketitle
\thispagestyle{empty}

\begin{center}
-- WORK IN PROGRESS -- \\
PLEASE DO NOT CITE OR REDISTRIBUTE WITHOUT PERMISSION
\end{center} 

\begin{abstract}\singlespacing
\noindent Despite scientific consensus about the imminent existential threat of human-induced climate change, misinformation on its cause, extent, and timeline is still common. Previous research has found that partisan bias is a major motivation to produce and spread misinformation, especially for such politically polarized issues. Thus, manipulation of partisan cues to correct falsehoods has been the subject of prior experimental work. However, non-partisan "expert" cues are a common source of factual information on climate change. Yet, there has been less attention to understanding how expert cues can reduce misinformation. We propose that factual expert cues can combat misinformation on politically contentious issues like climate change. We contend that experts delivering messages are more effective than impersonally providing facts, even without the co-partisan activation mechanism scrutinized in past research. Using a preregistered survey experiment, we present subjects with four expert cues: journalist, city manager, scientist, and activist. We expect that these expert cues will perform two functions to reduce misinformation about climate change: update respondents' beliefs with factual information, and increase the willingness of respondents to share the factual information. The influence of each expert cue can vary depending on an individual's prior belief and trust in the source cue. Therefore, we also examine how the expert's identities interact with race and gender.

\vspace{\baselineskip}
\noindent \textit{Keywords}: misinformation, elite cue, climate change, survey experiment 

\vspace{\baselineskip}
%\noindent \textit{Word count}: 5873 (via TeXcount)
\end{abstract}
\hfill

% \onehalfspacing
\doublespacing

\newpage\setcounter{page}{1}
\input{main}
\graphicspath{ {./images/} }
\clearpage
\singlespacing
\bibliographystyle{apsr2006}
\bibliography{Lab}

% \clearpage
% \singlespacing
% \renewcommand\thesubsection{\Roman{subsection}}
% \counterwithin{figure}{section}
% \counterwithin{table}{section}
% \setcounter{page}{1}
% \appendices
% \input{appendices}

\end{document}

\graphicspath{ {./images/} }
\clearpage
\singlespacing
\bibliographystyle{apsr2006}
\bibliography{Lab}

% \clearpage
% \singlespacing
% \renewcommand\thesubsection{\Roman{subsection}}
% \counterwithin{figure}{section}
% \counterwithin{table}{section}
% \setcounter{page}{1}
% \appendices
% \input{appendices}

\end{document}

\graphicspath{ {./images/} }
\clearpage
\singlespacing
\bibliographystyle{apsr2006}
\bibliography{Lab}

% \clearpage
% \singlespacing
% \renewcommand\thesubsection{\Roman{subsection}}
% \counterwithin{figure}{section}
% \counterwithin{table}{section}
% \setcounter{page}{1}
% \appendices
% \input{appendices}

\end{document}

\graphicspath{ {./images/} }
\clearpage
\singlespacing
\bibliographystyle{apsr2006}
\bibliography{Lab}

% \clearpage
% \singlespacing
% \renewcommand\thesubsection{\Roman{subsection}}
% \counterwithin{figure}{section}
% \counterwithin{table}{section}
% \setcounter{page}{1}
% \appendices
% \input{appendices}

\end{document}
